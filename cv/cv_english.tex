\documentclass[xetex,11pt,a4paper]{moderncv}
\usepackage{xltxtra}
\usepackage{polyglossia}
\moderncvstyle{classic}
\moderncvcolor{black}
\usepackage[left=2.0cm,top=2.5cm,right=3cm,bottom=2cm]{geometry}
%\defbibheading{none}{}
%\usepackage{lingulista}
\usepackage[
    natbib=true,
    style=authoryear-ibid,
    maxnames=2,
    hyperref=true,
    sortcites=true,
    firstinits=true,
    bibstyle=authoryear,
    isbn=false,doi=false,
    maxbibnames=15,
    opcittracker=true,
    loccittracker=true,
    terseinits=true,
    sorting=ydnt,
    labelnumber=true,
    backend=bibtex
    ]{biblatex}
\input{/home/mattis/library/latex/tex/lingulist/bibstyle}
\usepackage{enumitem}
\setlength{\bibhang}{\leftmargin}
\addtolength{\bibhang}{0.5cm}
\defbibenvironment{bibliography}
{\list{}
{\setlength{\leftmargin}{\bibhang}%
\setlength{\itemindent}{-\bibhang}%
\setlength{\itemsep}{\bibitemsep}%
\setlength{\parsep}{\bibparsep}}}
{\endlist}
{\item}

\bibliography{/home/mattis/Dropbox/evobib/basic}

\setlength{\bibhang}{20pt}

\setmainlanguage{german}

\setmainfont[Mapping=tex-text]{Arial}
\setsansfont[Mapping=tex-text]{Arial}
\setmonofont{FreeMono}
\newfontfamily{\hana}{HAN NOM A}

\firstname{Johann-Mattis}
\familyname{List}
\title{Lebenslauf}
\address{Frankfurter Str. 17a}{35037 Marburg}
%\mobile{0170/5302738}
\phone{+49-6421-28-25075}
\email{mattis.list@uni-marburg.de}
\homepage{http://lingulist.de}
%\photo{inga.jpg}
%\renewcommand{\firstnamefont}{\fontsize{12}{14}\sffamily\mdseries\upshape}
%\renewcommand{\titlefont}{\Large\sffamily\mdseries\slshape}

\newlength{\mywidth}
\setlength{\mywidth}{14.25cm}

\makeatletter

%\renewcommand*{\section}{sm}{%
%  \par\addvspace{2.5ex}%
%  \phantomsection{}% reset the anchor for hyperrefs
%  \addcontentsline{toc}{section}{#2}%
%  \parbox[t]{\hintscolumnwidth}{\strut\raggedleft\raisebox{\baseletterheight}{\color{color1}\rule{\hintscolumnwidth}{0.95ex}}}%
%  \hspace{\separatorcolumnwidth}%
%  \parbox[t]{\maincolumnwidth}{\strut\sectionstyle{#2}}%
%  \par\nobreak\addvspace{1ex}\@afterheading}% to avoid a pagebreak after the heading


          \fancypagestyle{plain}{%
	  \fancyhead[r]{\addressfont\itshape\strut Seite\ \thepage\ von\ \pageref{LastPage}}
	  \fancyhead[l]{\footnotesize J.-M. List -- Lebenslauf und
	  Publikationsliste}
	    \fancyfoot[r]{ }
	    }
\renewcommand*{\cventry}[7][.25em]{%
  \cvitem[#1]{#2}{%
    {\bfseries#3}%
    \ifthenelse{\equal{#4}{}}{}{, {#4}}%
    \ifthenelse{\equal{#5}{}}{}{, #5}%
    \ifthenelse{\equal{#6}{}}{}{, #6}%
    .\strut%
    \ifx&#7&%
      \else{\newline{}\begin{minipage}[t]{\mywidth}\small#7\end{minipage}}\fi}}

\renewcommand*{\cvitem}[3][.25em]{%
  \begin{tabular}{@{}p{\hintscolumnwidth}@{\hspace{\separatorcolumnwidth}}p{\mywidth}@{}}%
    \raggedright\hintstyle{\normalsize #2} &{\normalsize #3}%
  \end{tabular}%
  \par\addvspace{#1}}


\renewcommand*{\cvdoubleitem}[5][.25em]{%
 \cvitem[#1]{#2}{%
   \begin{minipage}[t]{\mywidth}#3\end{minipage}%
   \hfill% fill of \separatorcolumnwidth
   \begin{minipage}[t]{\hintscolumnwidth}\raggedright\hintstyle{#4}\end{minipage}%
   \hspace*{\separatorcolumnwidth}%
   \begin{minipage}[t]{\doubleitemmaincolumnwidth}#5\end{minipage}}}

\RenewDocumentCommand{\section}{sm}{%
  \par\addvspace{2.5ex}%
  \phantomsection{}% reset the anchor for hyperrefs
  \addcontentsline{toc}{section}{#2}%
  \textbf{\normalsize #2}
  %\parbox[t]{\hintscolumnwidth}{\strut\raggedleft\raisebox{\baseletterheight}{\color{color1}\rule{\hintscolumnwidth}{2pt}}}%
  %\hspace{\separatorcolumnwidth}%
  %\parbox[t]{\maincolumnwidth}{\strut \textbf{#2}}%
  \par\nobreak\addvspace{1ex}\@afterheading}% to avoid a pagebreak after the heading       

\makeatother


\usepackage{zhspacing}
\setlength{\hintscolumnwidth}{2.5cm} 
%\setlength{\maincolumnwidth}{12cm}% if you want to change the width of the column with the dates
%\setlength{\makecvtitlenamewidth}{12.0cm}           % for the 'classic' style, if you want to force the width allocated to your name and avoid line breaks. be careful though, the length is normally calculated to avoid any overlap with your personal info; use this at your own typographical risks...
%\setlength{\listitemmaincolumnwidth}{12cm}
%\setlength{\listdoubleitemmaincolumnwidth}{12cm}
%\setlength{\doubleitemmaincolumnwidth}{12cm}
\setlength{\textwidth}{17cm}
\setlength{\linewidth}{6cm}
% pagestyle settings
%\usepackage{scrpage2}
%\pagestyle{scrheadings}
%\ihead{\scriptsize\em J.-M. List -- Lebenslauf}
%%\chead{}
%\ohead{\scriptsize\em page \pagemark\ of \pageref{LastPage}}
%\ifoot{}
%\cfoot{}
%\ofoot{}


%\fancypagestyle{plain}{%
%\fancyhf{} 
%\fancyhead[R]{\thepage}
%\fancyfoot{}
%\rewnewcommand{\headrulewidth=2pt}}

\renewcommand*{\namefont}{\bfseries\large}
\renewcommand*{\titlefont}{\bfseries\large}
\renewcommand*{\addressfont}{\small}
\renewcommand*{\quotefont}{\large\slshape}

\begin{document}
\pagestyle{plain}
%\defbibenvironment{bibliography}
%{\list{}
%{\setlength{\leftmargin}{0.75cm+\bibhang}%
%\setlength{\itemindent}{-0.75cm}%
%\setlength{\itemsep}{\bibitemsep}%
%\setlength{\parsep}{\bibparsep}}}
%{\endlist}
%{\item[ ]}


%\noindent\bf Johann-Mattis List -- Lebenslauf und Publikationsliste \\
%\noindent\rule{\textwidth}{0.2pt}
\textbf{Johann-Mattis List -- Lebenslauf und Publikationsliste}\\ 
\noindent\begin{tabbing}
    gener.formation \= and \kill
    \footnotesize Geburtstag:                      \> 16.07.1981 \\
    \footnotesize Geburtsort:                      \> Kassel \\
    \footnotesize Adresse:                         \> Frankfurter Str. 17a \\
    \footnotesize                                  \> 35037 Marburg \\
    \footnotesize Telephon:                        \> +49-6421-28-25075 \\
    \footnotesize Email:                           \> \url{mattis.list@uni-marburg.de} \\
    \footnotesize Website:\> \url{http://lingulist.de}\\
\end{tabbing}
\section{Ausbildung}
%\cventry{seit 10/2012}{Wissenschaftlicher Mitarbeiter (Post-Doc)}{Ph
\cventry{{\footnotesize 06/02/2013}}{Promotion}{Heinrich Heine Universität}{Düsseldorf}{\textit{summa cum laude}}{Promotionsprüfung
(Disputatio) im Fach Allgemeine Linguistik.}{}
\cventry{{\footnotesize 02/2009--08/2012}}{Promotionsstudium}{Heinrich Heine
Universität}{Düsseldorf}{}{Vorbereitung der Promotion im Fach Allgemeine
Sprachwissenschaft mit dem Thema ,,Sequence Comparison in Historical
Linguistics".}{}
%\cventry{10/2008--01/2008}{Promotionsstudium}{Freie
%Universität}{Berlin}{}{Vorbereitung der Promotion im Fach Sinologie mit dem
%Thema ,,A Phonological Analysis of Yáng Xióng's {\em Fāngyán}".}{}
\cventry{{\footnotesize 25/06/2008}}{Magister Artium}{Freie Universität /
Humboldt-Universität}{Berlin}{\textit{1,0}}{Magisterprüfung in den Fächern Vergleichende und
Indogermanische Sprachwissenschaft (Hauptfach), Sinologie (Nebenfach) und
Russistik (Nebenfach).}{}
\cventry{{\footnotesize 09/2007--01/2008}}{Gaststudium}{Fudan
Universität}{Shanghai}{}{Belegung von Kursen der historischen Linguistik im Fach Allgemeine und Angewandte
Linguistik am Institut für chinesische Sprache und Literatur ({\hana 中國語言文學系}).}
\cventry{{\footnotesize 09/2005--07/2006}}{Sprachstudium}{Fudan
Universität}{Shanghai}{}{Belegung von chinesischen Sprachkursen
(fortgeschrittenes Niveau).}{}
\cventry{{\footnotesize 10/2003--09/2008}}{Magisterstudium}{Freie Universität /
Humboldt-Universität}{Berlin}{}{Studium der Fächer Vergleichende und
Indogermanische Sprachwissenschaft (Hauptfach), Sinologie (Nebenfach) und
Russistik (Nebenfach).}{}
\cventry{{\footnotesize 10/2002--09/2003}}{Magisterstudium}{Eberhard-Karls
Universität}{Tübingen}{}{Studium der Fächer Rhetorik (Hauptfach), Komparatistik
(Nebenfach), Geschichte (Nebenfach), Russistik (Nebenfach).}{}
\cventry{{\footnotesize 19/06/2001}}{Abitur}{Gymnasium Carolinum}{Osnabrück}{\textit{1,6}}{Abiturprüfungen in den
Prüfungsfächern Latein, Chemie, Geschichte und Musik}{}
\cventry{{\footnotesize 1994--2001}}{Sekundarstufe}{Gymnasium Carolinum}{Osnabrück}{}{}{}
 
\section{Promotion}
\cvitem{{\footnotesize Titel}}{\em Sequence Comparison in Historical Linguistics} 
\cvitem{{\footnotesize Betreuer}}{Prof. Dr. Hans Geisler, Dr. Wiebke Petersen}
\cvitem{{\footnotesize Beschreibung}}{\small In der Arbeit werden neue Algorithmen zur automatischen Alinierung und
Kognatenerkennung in der historischen Linguistik entwickelt, beschrieben und vorgestellt.}

\section{Magisterarbeit}
\cvitem{{\footnotesize Titel}}{\em Rekonstruktion der Aussprache des Mittel- und
Altchinesischen. Vergleich der Rekonstruktionsmethoden der indogermanischen und
der chinesischen Sprachwissenschaft} 
\cvitem{{\footnotesize Betreuer}}{Prof. Dr. Michael Meier-Brügger, Dr. Ingo Schäfer}
\cvitem{{\footnotesize Beschreibung}}{\small In der Arbeit werden die Methoden, welche zur
Rekonstruktion schriftlich unbelegter Sprachstufen auf Basis des internen und
externen Sprachvergleichs in der indogermanischen und der chinesischen
Sprachwissenschaft zur Anwendung kommen, beschrieben, untersucht und
verglichen.}
\section{Forschung}
\cventry{{\footnotesize seit 10/2012}}{Wissenschaftlicher Angestellter}{Philipps-Universität}{Marburg}{}{Mitarbeiter
(Post-Doktorand)
in der vom European Research Council geförderten Forschungseinheit ,,Quantitative Language
Comparison" (QuantHistLing) am
Forschungszentrum Deutscher Sprachatlas.}
\cventry{{\footnotesize 02/2009--09/2012}}{Wissenschaftlicher Angestellter}{Heinrich Heine
Universität}{Düsseldorf}{}{Mitarbeiter (Doktorand)
im vom Bundesministerium für Bildung und
Forschung geförderten Forschungsprojekt ,,Klassifikation und
Evolution in Biologie, Linguistik und Wissenschaftsgeschichte (EvoKlass)".}
\section{Lehre}
\noindent\textit{Summer Term 2014}\par\nopagebreak\vspace{0.25cm}
\nopagebreak\noindent List, J.-M. (2014): \textbf{Lautwandel} \ [Sound Change]. Institut für Sprache und Information, Heinrich Heine Universität Düsseldorf.\vspace{0.25cm}
\par
\nopagebreak\noindent List, J.-M. (2014): \textbf{Chinesische Dialektologie} \ [Chinese Dialectology]. Forschungszentrum Deutscher Sprachatlas, Philipps-Universität Marburg.\vspace{0.25cm}
\par
\noindent\textit{Winter Term 2013}\par\nopagebreak\vspace{0.25cm}
\nopagebreak\noindent List, J.-M. (2013): \textbf{Theoretische und praktische Aspekte der quantitativen historischen Linguistik} \ [Theoretical and practical aspects of quantitative historical linguistics]. Forschungszentrum Deutscher Sprachatlas, Philipps-Universität Marburg.\vspace{0.25cm}
\par
\noindent\textit{Summer Term 2013}\par\nopagebreak\vspace{0.25cm}
\nopagebreak\noindent List, J.-M. (2013): \textbf{Phonologie der chinesischen Dialekte} \ [Phonology of Chinese dialects]. Institut für Sprache und Information, Heinrich Heine Universität Düsseldorf.\vspace{0.25cm}
\par
\nopagebreak\noindent Cysouw, M. and J.-M. List (2013): \textbf{Digital Humanities: Einführung in die maschinelle Sprachverarbeitung} \ [Digital humanities: Introduction to natural language processing]. Forschungszentrum Deutscher Sprachatlas, Philipps-Universität Marburg.\vspace{0.25cm}
\par
\noindent\textit{Winter Term 2012}\par\nopagebreak\vspace{0.25cm}
\nopagebreak\noindent List, J.-M. (2012): \textbf{Neue Ansätze in der historischen Sprachwissenschaft} \ [New approaches in historical linguistics]. Institut für Sprache und Information, Heinrich Heine Universität Düsseldorf.\vspace{0.25cm}
\par
\noindent\textit{Winter Term 2011}\par\nopagebreak\vspace{0.25cm}
\nopagebreak\noindent List, J.-M. (2011): \textbf{LaTeX II} . Institut für Sprache und Information, Heinrich Heine Universität Düsseldorf.\vspace{0.25cm}
\par
\nopagebreak\noindent List, J.-M. (2011): \textbf{LaTeX I} . Institut für Sprache und Information, Heinrich Heine Universität Düsseldorf.\vspace{0.25cm}
\par
\nopagebreak\noindent List, J.-M. (2011): \textbf{Geschichte der Sprachwissenschaft} \ [History of Linguistics]. Institut für Sprache und Information, Heinrich Heine Universität Düsseldorf.\vspace{0.25cm}
\par
\noindent\textit{Summer Term 2011}\par\nopagebreak\vspace{0.25cm}
\nopagebreak\noindent List, J.-M. (2011): \textbf{Python} . Institut für Sprache und Information, Heinrich Heine Universität Düsseldorf.\vspace{0.25cm}
\par
\noindent\textit{Winter Term 2010}\par\nopagebreak\vspace{0.25cm}
\nopagebreak\noindent List, J.-M. (2010): \textbf{Chinesische Phonologie} \ [Chinese Phonology]. Institut für Sprache und Information, Heinrich Heine Universität Düsseldorf.\vspace{0.25cm}
\par
\noindent\textit{Summer Term 2010}\par\nopagebreak\vspace{0.25cm}
\nopagebreak\noindent List, J.-M. (2010): \textbf{Sprachkontakt und Sprachwandel} \ [Language contact and language change]. Institut für Sprache und Information, Heinrich Heine Universität Düsseldorf.\vspace{0.25cm}
\par
\noindent\textit{Winter Term 2009}\par\nopagebreak\vspace{0.25cm}
\nopagebreak\noindent List, J.-M. (2009): \textbf{Linguistische Rekonstruktion: Theorien und Methoden} \ [Linguistic Reconstruction: Theory and Methods]. Forschungszentrum Deutscher Sprachatlas, Philipps-Universität Marburg.\vspace{0.25cm}
\par
\noindent\textit{Summer Term 2009}\par\nopagebreak\vspace{0.25cm}
\nopagebreak\noindent Sommerfeld, S. and J.-M. List (2009): \textbf{Quantitative Methoden in der Klassifikation von Sprachen} \ [Quantitative Methods in Language Classification]. Institut für Sprache und Information, Heinrich Heine Universität Düsseldorf.\vspace{0.25cm}
\par

%\cventry{{\footnotesize SoSe 2014}}{Seminar}{Philipps-Universität}{Marburg}{}{Seminar ,,Chinesische
%Dialektologie", gehalten am Fachbereich für Germanistik und Kunstwissenschaften.}{}{}
%
%\cventry{{\footnotesize SoSe 2014}}{Seminar}{Heinrich Heine Universität}{Düsseldorf}{}{Seminar ,,Lautwandel",
%gehalten am Institut für Sprache und Information.}{}{}
%
%\cventry{{\footnotesize WiSe 2013}}{Seminar}{Philipps-Universität}{Marburg}{}{Seminar ,,Theoretische und praktische
%Aspekte der quantitativen historischen Linguistik", gehalten am Fachbereich für Germanistik und Kunstwissenschaften.}{}{}
%\cventry{{\footnotesize SoSe 2013}}{Seminar}{Philipps-Universität}{Marburg}{}{Seminar
%,,Digital Humanities: Einführung in die maschinelle Sprachverarbeitung" (zusammen mit M. Cysouw),
%gehalten am Fachbereich für Germanistik und Kunstwissenschaften.}{}{}
%\cventry{{\footnotesize SoSe 2013}}{Seminar}{Heinrich Heine Universität}{Düsseldorf}{}{Seminar ,,Phonologie der
%chinesischen Dialekte -- Synchrone und diachrone Aspekte", gehalten am Institut für Sprache und
%Information.}{}{}
%\cventry{{\footnotesize WiSe 2012}}{Seminar}{Heinrich Heine Universität}{Düsseldorf}{}{Seminar ,,Neue Ansätze in der
%historischen Linguistik", gehalten am Institut für Sprache und Information.}{}{}
%\cventry{{\footnotesize WiSe 2011}}{Doktorandenseminar}{Heinrich Heine Universität}{Düsseldorf}{}{,,LaTeX I und II",
%gehalten am Institut für Sprache und Information im Rahmen des SFB 991.}{}{}
%\cventry{{\footnotesize WiSe 2011}}{Seminar}{Heinrich Heine Universität}{Düsseldorf}{}{Seminar ,,Geschichte der
%Sprachwissenschaft", gehalten am Institut für Sprache und Information.}{}{}
%\cventry{{\footnotesize SoSe 2011}}{Seminar}{Heinrich Heine Universität}{Düsseldorf}{}{Seminar ,,Python", gehalten
%am Institut für Sprache und Information.}{}{}
%\cventry{{\footnotesize WiSe 2010}}{Seminar}{Heinrich Heine Universität}{Düsseldorf}{}{Seminar ,,Chinesische
%Phonologie", gehalten am Institut für Sprache und Information.}{}{}
%\cventry{{\footnotesize SoSe 2010}}{Seminar}{Heinrich Heine Universität}{Düsseldorf}{}{Seminar ,,Sprachkontakt und
%Sprachwandel", gehalten am Institut für Sprache und Information.}{}{}
%\cventry{{\footnotesize WiSe 2009}}{Seminar}{Heinrich Heine Universität}{Düsseldorf}{}{Seminar ,,Linguistische
%Rekonstruktion: Theorien und Methoden", gehalten am Institut für Sprache und Information.}{}{}
\section{Betreuung von Bachelor- und Masterarbeiten}
\cventry{{\footnotesize 11/2013}}{Marisa Delz, Masterarbeit ,,A theoretical approach on automatic loanword detection"}{Eberhard-Karls Universität}{Tübingen}{Zweitprüfer}{}
\cventry{{\footnotesize 06/2013}}{Yang Liu, Bachelorarbeit ,,Variation der Numeralklassifikation im Chinesischen"}{Philipps-Universität}{Marburg}{Zweitprüfer}{}
\cventry{{\footnotesize 10/2010}}{Isabell Haller, Bachelorarbeit ,,Modelle der genetischen Sprachklassifikation: Bäume, Wellen, Netze"}{Heinrich Heine Universität}{Düsseldorf}{Erstprüfer}{}
\section{Organisation}
\cventry{{\footnotesize 06/02/2014}}{Demo- und Postersession der Sektion Computerlinguistik im Rahmen
der 36. Tagung der Deutschen Gesellschaft für
Sprachwissenschaft}{Philipps-Universität}{Marburg}{}{Teilnehmer im Organisationsteam, zusammen mit Thomas Mayer und
Nathalia Levshina.}{}{}
\section{Sonstige Tätigkeiten}
\cventry{{\footnotesize 09/2001--08/2002}}{Zivildienst}{Kinderzirkus
Upsala}{Sankt Petersburg}{}{Mitarbeiter im sozialen Projekt ,,Kinderzirkus
Upsala"}
\section{Stipendien und Preise}
\cventry{{\footnotesize 03/02/2014}}{Beste Dissertation der Philosophischen Fakultät für das Jahr 2013}{Heinrich
Heine Universität}{Düsseldorf}{}{Mit 2500 Euro dotierter Preis, der jährlich von der philosophischen
Fakultät vergeben wird.}{}
\cventry{{\footnotesize 11-13/08/2013}}{Young Scholars Symposium}{University of
Washington}{Seattle}{}{Einladung der
``LFK
Society for Chinese Linguistics" zur voll finanzierten Teilnahme an dem LFK-Nachwuchssymposium.}{ }
\cventry{{\footnotesize 09/2007--01/2008}}{Semesterstipendium}{Deutscher Akademischer
Austauschdienst}{Bonn}{}{Stipendium für ein halbjähriges Gaststudium an der Fudan
Universität Shanghai.}{}
\cventry{{\footnotesize 09/2005--08/2006}}{Austauschstipendium}{Freie Universität Berlin / Fudan
Universität Shanghai}{Berlin / Shanghai}{}{Stipendium der Freien
Universität Berlin in Kooperation mit der Fudan Universität Shanghai für ein
einjähriges Chinesischstudium an der Fudan Universität Shanghai.}{}
%\cventry{{\footnotesize 20.08.2005}}{Lilalu Zirkuspreis 2005}{}{München}{\em Zweiter
%Platz in der Kategorie ,,Erwachsene / Amateure"}{}{}
\cventry{{\footnotesize 23/10/2004}}{Nachwuchsförderwettbewerb
Kleinkunst}{Sophienhof}{Kiel}{\em Zweiter Platz}{Mit 500 Euro dotierter zweiter Preis für komische
Ringjonglage.}{}
\cventry{{\footnotesize 07/06/2001}}{Bundeswettbewerb ,,Jugend musiziert"}{Bundesministerium
für Familie, Senioren, Frauen und Jugend}{Hamburg}{Zweiter Preis}{\em Zweiter Preis
verliehen in der Kategorie ,,Zupfinstrumente / Trio" (zusammen mit Katun Luc und Katharina
Dornieden.}{}{}{}
 
\section{Sprachkenntnisse}
\cvlanguage{{\footnotesize Deutsch}}{Muttersprache}{}
\cvlanguage{{\footnotesize Russisch}}{verhandlungssicher}{}
\cvlanguage{{\footnotesize Chinesisch}}{verhandlungssicher} {HSK: B (7)\textcolor{white}{....}}
\cvlanguage{{\footnotesize English}}{verhandlungssicher}{TOEFL iBT: 107\textcolor{white}{....}}
\cvlanguage{{\footnotesize Latein}}{gute Kenntnisse}    {Großes Latinum\textcolor{white}{....}}
\cvlanguage{{\footnotesize Altgriechisch}}{gute Kenntnisse}    {Graecum\textcolor{white}{....}}
\cvlanguage{{\footnotesize Sanskrit}}{gute Kenntnisse}{}
\cvlanguage{{\footnotesize Schwedisch}}{Grundkenntnisse}{}
\cvlanguage{{\footnotesize Dänisch}}{Grundkenntnisse}{}
\cvlanguage{{\footnotesize Französisch}}{Grundkenntnisse}{}
 
\section{Computerkenntnisse}
\cvlanguage{{\footnotesize Sprachen}}{Python, R, C{\ttfamily ++}, AWK, BASH (Unix Shell), PHP,
SQL, JavaScript}{}
\cvlanguage{{\footnotesize Formatierung}}{\LaTeX{},
Microsoft Word, LibreOffice}{}
\cvlanguage{{\footnotesize Grafik}}{\hspace{0.0cm}PSTricks, GIMP}{}
\cvlanguage{{\footnotesize
Betriebssysteme}}{\hspace{0.0cm}Linux, Windows}{}

\section{Publikationen}


\subsection{Papers to appear}
 
\begin{itemize}
\item[] Geisler, Hans and List, Johann-Mattis (forthcoming): \textbf{Beautiful trees on unstable ground. Notes on the data problem in lexicostatistics}. In Hettrich, Heinrich (ed)  \textit{Die Ausbreitung des Indogermanischen. Thesen aus Sprachwissenschaft, Archäologie und Genetik} Reichert: Wiesbaden. .

\end{itemize}
\subsection{Papers from 2014}
 
\begin{itemize}
\item[] List, Johann-Mattis; Nelson-Sathi, Shijulal; Martin, William and Geisler, Hans (2014): \textbf{Using phylogenetic networks to model Chinese dialect history}. \textit{Language Dynamics and Change}} 4.2. . 222–252.

\item[] List, J.-M. (2014): \textbf{Sequence comparison in historical linguistics}. Düsseldorf University Press: Düsseldorf.

\item[] Mayer, Thomas; List, Johann-Mattis; Terhalle, Anselm and Urban, Matthias (2014): \textbf{An interactive visualization of cross-linguistic colexification patterns}. In \textit{Visualization as added value in the development, use and evaluation of Linguistic Resources. Workshop organized as part of the International Conference on Language Resources and Evaluation}. . 1-8.

\item[] List, J.-M. and Prokić, J. (2014): \textbf{A benchmark database of phonetic alignments in historical linguistics and dialectology.}. In \textit{Proceedings of the Ninth International Conference on Language Resources and Evaluation}. . 288-294.

\item[] List, J.-M. (2014): \textbf{Investigating the impact of sample size on cognate detection}. \textit{Journal of Language Relationship}} 11. . 91-101.

\item[] List, Johann-Mattis; Nelson-Sathi, Shijulal; Geisler, Hans and Martin, William (2014): \textbf{Networks of lexical borrowing and lateral gene transfer in language and genome evolution}. \textit{Bioessays}} 36.2. . 141-150.

\end{itemize}
\subsection{Papers from 2013}
 
\begin{itemize}
\item[] Shijulal Nelson-Sathi; Popa, Ovidiu; List, Johann-Mattis; Geisler, Hans; Martin, William F. and Dagan, Tal (2013): \textbf{Reconstructing the lateral component of language history and genome evolution using network approaches}. In Heiner Fangerau et al. (eds)  \textit{Classification and evolution in biology, linguistics and the history of science. Concepts – methods – visualization.} Franz Steiner Verlag: Stuttgart. . 163-180.

\item[] Lopez, P.; List, J.-M. and Bapteste, E. (2013): \textbf{A preliminary case for exploratory networks in biology and linguistics: the phonetic network of Chinese words as a case-study}. In Heiner Fangerau et al. (eds)  \textit{Classification and evolution in biology, linguistics and the history of science. Concepts – methods – visualization.} Franz Steiner Verlag: Stuttgart. . 181-196.

\item[] Geisler, H. and List, J.-M. (2013): \textbf{Do languages grow on trees? The tree metaphor in the history of linguistics}. In Heiner Fangerau et al. (eds)  \textit{Classification and evolution in biology, linguistics and the history of science. Concepts – methods – visualization.} Franz Steiner Verlag: Stuttgart. . 111-124.

\item[] List, Johann-Mattis and Moran, Steven (2013): \textbf{An open source toolkit for quantitative historical linguistics}. In \textit{Proceedings of the ACL 2013 System Demonstrations}. . 13-18.

\item[] List, Johann-Mattis; Terhalle, Anselm and Urban, Matthias (2013): \textbf{Using network approaches to enhance the analysis of cross-linguistic polysemies}. In \textit{Proceedings of the 10th International Conference on Computational Semantics -- Short Papers}. . 347-353.

\end{itemize}
\subsection{Papers from 2012}
 
\begin{itemize}
\item[] List, Johann-Mattis (2012): \textbf{Improving phonetic alignment by handling secondary sequence structures}. In \textit{Computational approaches to the study of dialectal and typological variation. Working papers submitted for the workshop organized as part of the ESSLLI 2012}. .

\item[] List, Johann-Mattis (2012): \textbf{LexStat. Automatic detection of cognates in multilingual wordlists}. In \textit{Proceedings of the EACL 2012 Joint Workshop of Visualization of Linguistic Patterns and Uncovering Language History from Multilingual Resources}. . 117-125.

\item[] List, Johann-Mattis (2012): \textbf{SCA. Phonetic alignment based on sound classes}. In Slavkovik, Marija and Lassiter, Dan (eds)  \textit{New directions in logic, language, and computation} Springer: Berlin and Heidelberg. . 32-51.

\item[] List, Johann-Mattis (2012): \textbf{Multiple sequence alignment in historical linguistics. A sound class based approach}. In \textit{Proceedings of ConSOLE XIX}. . 241-260.

\end{itemize}
\subsection{Papers from 2011}
 
\begin{itemize}
\item[] Holman, Eric. W.; Brown, Cecil H.; Wichmann, Søren; M\"uller, Andr\'e; Velupillai, Viveka; Hammarstr\"om, Harald; Sauppe, Sebastian; Jung, Hagen; Bakker, Dik; Brown, Pamela; Belyaev, Oleg; Urban, Matthias; Mailhammer, Robert; List, Johann-Mattis and Egorov, Dimitry (2011): \textbf{Automated dating of the world's language families based on lexical similarity}. \textit{Current Anthropology}} 52.6. . 841-875.

\item[] Nelson-Sathi, Shijulal; List, Johann-Mattis; Geisler, Hans; Fangerau, Heiner; Gray, Russell D.; Martin, William and Dagan, Tal (2011): \textbf{Networks uncover hidden lexical borrowing in Indo-European language evolution}. \textit{Proceedings of the Royal Society B}} 278.1713. . 1794-1803.

\end{itemize}
\subsection{Papers from 2010}
 
\begin{itemize}
\item[] List, Johann-Mattis (2010): \textbf{Phonetic alignment based on sound classes. A new method for sequence comparison in historical linguistics}. In \textit{Proceedings of the 15th Student Session of the European Summer School for Logic, Language and Information}. . 192-202.

\item[] Wichmann, Søren; Holman, Eric W.; Müller, André; Velupillai, Viveka; List, Johann-Mattis; Belyaev, Oleg; Urban, Matthias and Bakker, Dik (2010): \textbf{Glottochronology as a heuristic for genealogical language relationships}. \textit{Journal of Quantitative Linguistics}} 17.4. . 303-316.

\end{itemize}
\subsection{Papers from 2009}
 
\begin{itemize}
\item[] List, Johann-Mattis (2009): \textbf{Sprachvariation im modernen Chinesisch}. \textit{CHUN -- Chinesischunterricht}} 24. . 123-140.

\end{itemize}
\subsection{Papers from 2008}
 
\begin{itemize}
\item[] List, J.-M. (2008): \textbf{The validity of reconstruction systems}. Research Proposal.

\item[] List, Johann-Mattis (2008): \textbf{The puzzling case of Boshan tone sandhi}. Working paper.

\item[] List, Johann-Mattis (2008): \textbf{Resultativkomposita im Chinesischen [Resultative compounds in Chinese]}. Summary.

\item[] List, Johann-Mattis (2008): \textbf{Rekonstruktion der Aussprache des Mittel- und Altchinesischen. Vergleich der Rekonstruktionsmethoden der indogermanischen und der chinesischen Sprachwissenschaft [Reconstruction of the pronunciation of Middle and Old Chinese. Comparison of reconstruction methods in Indo-European and Chinese linguistics]}. Magister thesis. Freie Universität Berlin.

\end{itemize}
\subsection{Papers from 2007}
 
\begin{itemize}
\item[] List, Johann-Mattis (2007): \textbf{Typology and linguistic reconstruction}. Working paper.

\item[] List, Johann-Mattis (2007): \textbf{The derivational character of the Chinese writing system}. Working paper.

\item[] List, Johann-Mattis (2007): \textbf{Osnovy rekonstrukcii srednekitajskogo i drevnekitajskogo jazykov [Introduction to the reconstruction of Middle and Old Chinese]}. Summary.

\item[] List, Johann-Mattis (2007): \textbf{\textit{Cóng Éwén de jiǎodù lái kàn pànduàn dòngcí "shì" de qǐyuán yǔ yǎnbiàn} 从俄文的角度来看判断动词“是”的起源与演变 [Origin and change of the Chinese copula shì from a Russian perspective]}. Working paper.

\end{itemize}
\end{document}
