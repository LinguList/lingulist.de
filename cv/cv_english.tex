\documentclass[xetex,11pt,a4paper]{moderncv}
\usepackage{xltxtra}
\usepackage{polyglossia}
\moderncvstyle{classic}
\moderncvcolor{black}
\usepackage[left=2.0cm,top=2.5cm,right=3cm,bottom=2cm]{geometry}
%\defbibheading{none}{}
\usepackage{lastpage}
%\usepackage{lingulista}
\usepackage[
    natbib=true,
    style=authoryear-ibid,
    maxnames=2,
    hyperref=true,
    sortcites=true,
    firstinits=true,
    bibstyle=authoryear,
    isbn=false,doi=false,
    maxbibnames=15,
    opcittracker=true,
    loccittracker=true,
    terseinits=true,
    sorting=ydnt,
    labelnumber=true,
    backend=bibtex
    ]{biblatex}
\input{/home/mattis/library/latex/tex/lingulist/bibstyle}
\usepackage{enumitem}
\setlength{\bibhang}{\leftmargin}
\addtolength{\bibhang}{0.5cm}
\defbibenvironment{bibliography}
{\list{}
{\setlength{\leftmargin}{\bibhang}%
\setlength{\itemindent}{-\bibhang}%
\setlength{\itemsep}{\bibitemsep}%
\setlength{\parsep}{\bibparsep}}}
{\endlist}
{\item}


\setlength{\bibhang}{20pt}

\setmainlanguage{german}

\setmainfont[Mapping=tex-text]{Arial}
\setsansfont[Mapping=tex-text]{Arial}
\setmonofont{FreeMono}
\newfontfamily{\hana}{HAN NOM A}

\firstname{Johann-Mattis}
\familyname{List}
\title{Curriculumn Vitae}
\address{Frankfurter Str. 17a}{35037 Marburg}
%\mobile{0170/5302738}
\phone{+49-6421-28-25075}
\email{mattis.list@uni-marburg.de}
\homepage{http://lingulist.de}
%\photo{inga.jpg}
%\renewcommand{\firstnamefont}{\fontsize{12}{14}\sffamily\mdseries\upshape}
%\renewcommand{\titlefont}{\Large\sffamily\mdseries\slshape}

\newlength{\mywidth}
\setlength{\mywidth}{14.25cm}

\makeatletter

%\renewcommand*{\section}{sm}{%
%  \par\addvspace{2.5ex}%
%  \phantomsection{}% reset the anchor for hyperrefs
%  \addcontentsline{toc}{section}{#2}%
%  \parbox[t]{\hintscolumnwidth}{\strut\raggedleft\raisebox{\baseletterheight}{\color{color1}\rule{\hintscolumnwidth}{0.95ex}}}%
%  \hspace{\separatorcolumnwidth}%
%  \parbox[t]{\maincolumnwidth}{\strut\sectionstyle{#2}}%
%  \par\nobreak\addvspace{1ex}\@afterheading}% to avoid a pagebreak after the heading


          \fancypagestyle{plain}{%
	  \fancyhead[r]{\addressfont\itshape\strut Page\ \thepage\ of\ \pageref{LastPage}}
	  \fancyhead[l]{\footnotesize J.-M. List -- Curriculumn Vitae und
	  Publication List}
	    \fancyfoot[r]{ }
	    }
\renewcommand*{\cventry}[7][.25em]{%
  \cvitem[#1]{#2}{%
    {\bfseries#3}%
    \ifthenelse{\equal{#4}{}}{}{, {#4}}%
    \ifthenelse{\equal{#5}{}}{}{, #5}%
    \ifthenelse{\equal{#6}{}}{}{, #6}%
    .\strut%
    \ifx&#7&%
      \else{\newline{}\begin{minipage}[t]{\mywidth}\small#7\end{minipage}}\fi}}

\renewcommand*{\cvitem}[3][.25em]{%
  \begin{tabular}{@{}p{\hintscolumnwidth}@{\hspace{\separatorcolumnwidth}}p{\mywidth}@{}}%
    \raggedright\hintstyle{\normalsize #2} &{\normalsize #3}%
  \end{tabular}%
  \par\addvspace{#1}}


\renewcommand*{\cvdoubleitem}[5][.25em]{%
 \cvitem[#1]{#2}{%
   \begin{minipage}[t]{\mywidth}#3\end{minipage}%
   \hfill% fill of \separatorcolumnwidth
   \begin{minipage}[t]{\hintscolumnwidth}\raggedright\hintstyle{#4}\end{minipage}%
   \hspace*{\separatorcolumnwidth}%
   \begin{minipage}[t]{\doubleitemmaincolumnwidth}#5\end{minipage}}}

\RenewDocumentCommand{\section}{sm}{%
  \par\addvspace{2.5ex}%
  \phantomsection{}% reset the anchor for hyperrefs
  \addcontentsline{toc}{section}{#2}%
  \textbf{\normalsize #2}
  %\parbox[t]{\hintscolumnwidth}{\strut\raggedleft\raisebox{\baseletterheight}{\color{color1}\rule{\hintscolumnwidth}{2pt}}}%
  %\hspace{\separatorcolumnwidth}%
  %\parbox[t]{\maincolumnwidth}{\strut \textbf{#2}}%
  \par\nobreak\addvspace{1ex}\@afterheading}% to avoid a pagebreak after the heading       

\makeatother


\usepackage{zhspacing}
\setlength{\hintscolumnwidth}{2.5cm} 
%\setlength{\maincolumnwidth}{12cm}% if you want to change the width of the column with the dates
%\setlength{\makecvtitlenamewidth}{12.0cm}           % for the 'classic' style, if you want to force the width allocated to your name and avoid line breaks. be careful though, the length is normally calculated to avoid any overlap with your personal info; use this at your own typographical risks...
%\setlength{\listitemmaincolumnwidth}{12cm}
%\setlength{\listdoubleitemmaincolumnwidth}{12cm}
%\setlength{\doubleitemmaincolumnwidth}{12cm}
\setlength{\textwidth}{17cm}
\setlength{\linewidth}{6cm}
% pagestyle settings
%\usepackage{scrpage2}
%\pagestyle{scrheadings}
%\ihead{\scriptsize\em J.-M. List -- Lebenslauf}
%%\chead{}
%\ohead{\scriptsize\em page \pagemark\ of \pageref{LastPage}}
%\ifoot{}
%\cfoot{}
%\ofoot{}


%\fancypagestyle{plain}{%
%\fancyhf{} 
%\fancyhead[R]{\thepage}
%\fancyfoot{}
%\rewnewcommand{\headrulewidth=2pt}}

\renewcommand*{\namefont}{\bfseries\large}
\renewcommand*{\titlefont}{\bfseries\large}
\renewcommand*{\addressfont}{\small}
\renewcommand*{\quotefont}{\large\slshape}

\begin{document}
\pagestyle{plain}
%\defbibenvironment{bibliography}
%{\list{}
%{\setlength{\leftmargin}{0.75cm+\bibhang}%
%\setlength{\itemindent}{-0.75cm}%
%\setlength{\itemsep}{\bibitemsep}%
%\setlength{\parsep}{\bibparsep}}}
%{\endlist}
%{\item[ ]}


%\noindent\bf Johann-Mattis List -- Lebenslauf und Publikationsliste \\
%\noindent\rule{\textwidth}{0.2pt}
\textbf{Johann-Mattis List -- Curriculum Vitae and Publication List}\\ 
\noindent\begin{tabbing}
    gener.formation \= and \kill
    \footnotesize Date of Birth:                      \> 16.07.1981 \\
    \footnotesize Place of Birth:                      \> Kassel \\
    \footnotesize Address:                         \> Frankfurter Str. 17a \\
    \footnotesize                                  \> 35037 Marburg \\
    \footnotesize Telephone:                        \> +49-6421-28-25075 \\
    \footnotesize Email:                           \> \url{mattis.list@uni-marburg.de} \\
    \footnotesize Website:\> \url{http://lingulist.de}\\
\end{tabbing}
\section{Education}
%\cventry{seit 10/2012}{Wissenschaftlicher Mitarbeiter (Post-Doc)}{Ph
\cventry{{\footnotesize 06/02/2013}}{PhD}{Heinrich Heine Universität}{Düsseldorf}{\textit{summa cum
laude}}{Disputatio in General Linguistics}{}
\cventry{{\footnotesize 02/2009--08/2012}}{Doctoral Studies}{Heinrich Heine
Universität}{Düsseldorf}{}{Topic ``Sequence Comparison in Historical
Linguistics".}{}
%\cventry{10/2008--01/2008}{Promotionsstudium}{Freie
%Universität}{Berlin}{}{Vorbereitung der Promotion im Fach Sinologie mit dem
%Thema ``A Phonological Analysis of Yáng Xióng's {\em Fāngyán}".}{}
\cventry{{\footnotesize 25/06/2008}}{Magister Artium}{Freie Universität /
Humboldt-Universität}{Berlin}{\textit{1,0}}{Master degree in Comparative and Indo-European
Linguistics (major) and Sinology and Russian Philology (minors).}{}
\cventry{{\footnotesize 09/2007--01/2008}}{Visiting Student}{Fúdàn
University}{Shànghǎi}{}{Courses in historical linguistics and General and Applied Lingusitics,
Institute for Chines Language and Literature ({\hana 中國語言文學系}).}
\cventry{{\footnotesize 09/2005--07/2006}}{Language Student}{Fúdàn
University}{Shànghǎi}{}{Chinese language courses (advanced level)}{}
\cventry{{\footnotesize 10/2003--09/2008}}{Master (Magister) Studies}{Freie Universität /
Humboldt-Universität}{Berlin}{}{Comparative and Indo-European
Linguistics (major) and Sinology and Russian Philology (minors).}{}
\cventry{{\footnotesize 10/2002--09/2003}}{Master (Magister) Studies}{Eberhard-Karls
Universität}{Tübingen}{}{Rhetorics (major), Comparatistics and Russian Philology (minors).}{}
\cventry{{\footnotesize 19/06/2001}}{Abitur (Highschool Diploma)}{Gymnasium
Carolinum}{Osnabrück}{\textit{1.6}}{Highschool Diploma, with Latin, Chemistry, History, and Music as
main subjects.}{}
 
\section{PhD Thesis}
\cvitem{{\footnotesize Title}}{\em Sequence Comparison in Historical Linguistics} 
\cvitem{{\footnotesize Supervisors}}{Prof. Dr. Hans Geisler, Dr. Wiebke Petersen}
\cvitem{{\footnotesize Description}}{\small The thesis introduces new methods for automatic phonetic alignment and automatic cognate detection.}

\section{Magister Thesis}
\cvitem{{\footnotesize Title}}{\em Rekonstruktion der Aussprache des Mittel- und
Altchinesischen. Vergleich der Rekonstruktionsmethoden der indogermanischen und
der chinesischen Sprachwissenschaft {\em [Reconstruction of the pronunciation of Middle and Old Chinese.
Comparison of the reconstruction methods in Indo-European and Chinese linguistics]}} 
\cvitem{{\footnotesize Supervisors}}{Prof. Dr. Michael Meier-Brügger, Dr. Ingo Schäfer}
\cvitem{{\footnotesize Description}}{\small The thesis deals with the methods that are applied in order to reconstruct unattested stages of ancestral
languages in Indo-European and Chinese linguistics..}
\section{Research}
\cventry{{\footnotesize since 10/2012}}{Research Assistant}{Philipps-Universität}{Marburg}{}{Research
assistant (post-doc) in the ERC-funded research project ``Quantitative Language Comparison" at the
Research Center Deutscher Sprachatlas.}
\cventry{{\footnotesize 02/2009--09/2012}}{Research Assistant}{Heinrich Heine
Universität}{Düsseldorf}{}{Research assistant (doctoral student) in the BMBF-funded research project
``Classification and Evolution in Biology, Linguistics, and the History of Science".}
\section{Teaching}
\noindent\textit{Sommersemester 2014}\par\nopagebreak\vspace{0.25cm}
\nopagebreak\noindent List, J.-M. (2014): \textbf{Lautwandel}\ [Sound Change]. Institut für Sprache und InformationHeinrich Heine Universität Düsseldorf.\vspace{0.25cm}
\par
\nopagebreak\noindent List, J.-M. (2014): \textbf{Chinesische Dialektologie}\ [Chinese Dialectology]. Forschungszentrum Deutscher SprachatlasPhilipps-Universität Marburg.\vspace{0.25cm}
\par
\noindent\textit{Wintersemester 2013}\par\nopagebreak\vspace{0.25cm}
\nopagebreak\noindent List, J.-M. (2013): \textbf{Theoretische und praktische Aspekte der quantitativen historischen Linguistik}\ [Theoretical and practical aspects of quantitative historical linguistics]. Forschungszentrum Deutscher SprachatlasPhilipps-Universität Marburg.\vspace{0.25cm}
\par
\noindent\textit{Sommersemester 2013}\par\nopagebreak\vspace{0.25cm}
\nopagebreak\noindent List, J.-M. (2013): \textbf{Phonologie der chinesischen Dialekte}\ [Phonology of Chinese dialects]. Institut für Sprache und InformationHeinrich Heine Universität Düsseldorf.\vspace{0.25cm}
\par
\nopagebreak\noindent Cysouw, M. and J.-M. List (2013): \textbf{Digital Humanities: Einführung in die maschinelle Sprachverarbeitung}\ [Digital humanities: Introduction to natural language processing]. Forschungszentrum Deutscher SprachatlasPhilipps-Universität Marburg.\vspace{0.25cm}
\par
\noindent\textit{Wintersemester 2012}\par\nopagebreak\vspace{0.25cm}
\nopagebreak\noindent List, J.-M. (2012): \textbf{Neue Ansätze in der historischen Sprachwissenschaft}\ [New approaches in historical linguistics]. Institut für Sprache und InformationHeinrich Heine Universität Düsseldorf.\vspace{0.25cm}
\par
\noindent\textit{Wintersemester 2011}\par\nopagebreak\vspace{0.25cm}
\nopagebreak\noindent List, J.-M. (2011): \textbf{LaTeX II}. Institut für Sprache und InformationHeinrich Heine Universität Düsseldorf.\vspace{0.25cm}
\par
\nopagebreak\noindent List, J.-M. (2011): \textbf{LaTeX I}. Institut für Sprache und InformationHeinrich Heine Universität Düsseldorf.\vspace{0.25cm}
\par
\nopagebreak\noindent List, J.-M. (2011): \textbf{Geschichte der Sprachwissenschaft}\ [History of Linguistics]. Institut für Sprache und InformationHeinrich Heine Universität Düsseldorf.\vspace{0.25cm}
\par
\noindent\textit{Sommersemester 2011}\par\nopagebreak\vspace{0.25cm}
\nopagebreak\noindent List, J.-M. (2011): \textbf{Python}. Institut für Sprache und InformationHeinrich Heine Universität Düsseldorf.\vspace{0.25cm}
\par
\noindent\textit{Wintersemester 2010}\par\nopagebreak\vspace{0.25cm}
\nopagebreak\noindent List, J.-M. (2010): \textbf{Chinesische Phonologie}\ [Chinese Phonology]. Institut für Sprache und InformationHeinrich Heine Universität Düsseldorf.\vspace{0.25cm}
\par
\noindent\textit{Sommersemester 2010}\par\nopagebreak\vspace{0.25cm}
\nopagebreak\noindent List, J.-M. (2010): \textbf{Sprachkontakt und Sprachwandel}\ [Language contact and language change]. Institut für Sprache und InformationHeinrich Heine Universität Düsseldorf.\vspace{0.25cm}
\par
\noindent\textit{Wintersemester 2009}\par\nopagebreak\vspace{0.25cm}
\nopagebreak\noindent List, J.-M. (2009): \textbf{Linguistische Rekonstruktion: Theorien und Methoden}\ [Linguistic Reconstruction: Theory and Methods]. Forschungszentrum Deutscher SprachatlasPhilipps-Universität Marburg.\vspace{0.25cm}
\par
\noindent\textit{Sommersemester 2009}\par\nopagebreak\vspace{0.25cm}
\nopagebreak\noindent Sommerfeld, S. and J.-M. List (2009): \textbf{Quantitative Methoden in der Klassifikation von Sprachen}\ [Quantitative Methods in Language Classification]. Institut für Sprache und InformationHeinrich Heine Universität Düsseldorf.\vspace{0.25cm}
\par

%\cventry{{\footnotesize SoSe 2014}}{Seminar}{Philipps-Universität}{Marburg}{}{Seminar ``Chinesische
%Dialektologie", gehalten am Fachbereich für Germanistik und Kunstwissenschaften.}{}{}
%
%\cventry{{\footnotesize SoSe 2014}}{Seminar}{Heinrich Heine Universität}{Düsseldorf}{}{Seminar ``Lautwandel",
%gehalten am Institut für Sprache und Information.}{}{}
%
%\cventry{{\footnotesize WiSe 2013}}{Seminar}{Philipps-Universität}{Marburg}{}{Seminar ``Theoretische und praktische
%Aspekte der quantitativen historischen Linguistik", gehalten am Fachbereich für Germanistik und Kunstwissenschaften.}{}{}
%\cventry{{\footnotesize SoSe 2013}}{Seminar}{Philipps-Universität}{Marburg}{}{Seminar
%``Digital Humanities: Einführung in die maschinelle Sprachverarbeitung" (zusammen mit M. Cysouw),
%gehalten am Fachbereich für Germanistik und Kunstwissenschaften.}{}{}
%\cventry{{\footnotesize SoSe 2013}}{Seminar}{Heinrich Heine Universität}{Düsseldorf}{}{Seminar ``Phonologie der
%chinesischen Dialekte -- Synchrone und diachrone Aspekte", gehalten am Institut für Sprache und
%Information.}{}{}
%\cventry{{\footnotesize WiSe 2012}}{Seminar}{Heinrich Heine Universität}{Düsseldorf}{}{Seminar ``Neue Ansätze in der
%historischen Linguistik", gehalten am Institut für Sprache und Information.}{}{}
%\cventry{{\footnotesize WiSe 2011}}{Doktorandenseminar}{Heinrich Heine Universität}{Düsseldorf}{}{``LaTeX I und II",
%gehalten am Institut für Sprache und Information im Rahmen des SFB 991.}{}{}
%\cventry{{\footnotesize WiSe 2011}}{Seminar}{Heinrich Heine Universität}{Düsseldorf}{}{Seminar ``Geschichte der
%Sprachwissenschaft", gehalten am Institut für Sprache und Information.}{}{}
%\cventry{{\footnotesize SoSe 2011}}{Seminar}{Heinrich Heine Universität}{Düsseldorf}{}{Seminar ``Python", gehalten
%am Institut für Sprache und Information.}{}{}
%\cventry{{\footnotesize WiSe 2010}}{Seminar}{Heinrich Heine Universität}{Düsseldorf}{}{Seminar ``Chinesische
%Phonologie", gehalten am Institut für Sprache und Information.}{}{}
%\cventry{{\footnotesize SoSe 2010}}{Seminar}{Heinrich Heine Universität}{Düsseldorf}{}{Seminar ``Sprachkontakt und
%Sprachwandel", gehalten am Institut für Sprache und Information.}{}{}
%\cventry{{\footnotesize WiSe 2009}}{Seminar}{Heinrich Heine Universität}{Düsseldorf}{}{Seminar ``Linguistische
%Rekonstruktion: Theorien und Methoden", gehalten am Institut für Sprache und Information.}{}{}
\section{Supervisor of Bachelor and Master Thesises}
\cventry{{\footnotesize 11/2013}}{Marisa Delz, Master Thesis ``A theoretical approach on automatic
loanword detection"}{Eberhard-Karls Universität}{Tübingen}{Secondary Supervisor}{}
\cventry{{\footnotesize 06/2013}}{Yang Liu, Bachelor Thesis ``Variation der Numeralklassifikation im
Chinesischen [Variation of numeral classification in
Chinese]"}{Philipps-Universität}{Marburg}{Secondary Supervisor}{}
\cventry{{\footnotesize 10/2010}}{Isabell Haller, Bachelor Thesis ``Modelle der genetischen
Sprachklassifikation: Bäume, Wellen, Netze [Models of genetic language classification: Trees, Waves,
Networks]"}{Heinrich Heine Universität}{Düsseldorf}{Supervisor}{}
\section{Organisation}
\cventry{{\footnotesize 06/02/2014}}{Demo and poster session of the section on computational
linguistics organized as part of the 36th Meeting of the German Society for Linguistics (Deutsche
Gesellschaft für Sprachwissenschaft)}{Philipps-Universität}{Marburg}{}{Participant of the
organisational board, together with Thomas Mayer and
Nathalia Levshina.}{}{}
\section{Further Occupations and Activities}
\cventry{{\footnotesize 09/2001--08/2002}}{Civil Service}{Children's Circus
Upsala}{Saint Petersburg}{}{Assistant and juggling trainer in the social project ``Upsala Cirk
[Upsala Circus]".}
\section{Scholarships and Awards}
\cventry{{\footnotesize 03/02/2014}}{Best Dissertation of the Philosophical Faculty in 2013}{Heinrich
Heine Universität}{Düsseldorf}{}{Annual prize (2500 Euros) awarded by the Philosophical Faculty of
 Heinrich Heine Universität Düsseldorf.}{}
\cventry{{\footnotesize 11-13/08/2013}}{Young Scholars Symposium}{University of
Washington}{Seattle}{}{Invitation of the ``LFK
Society for Chinese Linguistics" to take part in the fully sponsored LFK Young Scholars Symposium.}{ }
\cventry{{\footnotesize 09/2007--01/2008}}{Semester Scholarship}{German Academic Exchange Service}{Bonn}{}{
Scholarship for one semester at Fúdàn University Shànghǎi.}{}
\cventry{{\footnotesize 09/2005--08/2006}}{Exchange Scholarship}{Freie Universität Berlin / Fúdàn
Universität Shanghai}{Berlin / Shanghai}{}{Scholarship of Freie
Universität Berlin in cooperation with Fúdàn University Shànghǎi for two semesters of Chinese
language studies at Fúdàn University Shànghǎi.}{}
%\cventry{{\footnotesize 20.08.2005}}{Lilalu Zirkuspreis 2005}{}{München}{\em Zweiter
%Platz in der Kategorie ``Erwachsene / Amateure"}{}{}
\cventry{{\footnotesize 23/10/2004}}{Young Talents of Cabaret Competition}{Sophienhof}{Kiel}{\em
Second Prize}{Second prize (500 Euros) for a ``Funny Juggling Performance with Rings".}{}
\cventry{{\footnotesize 07/06/2001}}{Federal Competition ``Jugend musiziert"}{Federal Ministry for Family Affairs, Senior
Citizens, Women and Youth}{Hamburg}{Second Prize}{\em Second Price, Category ``Plucked Instruments /
Trio" (together with Katun Luc and Katharina
Dornieden.}{}{}{}
 
\section{Language Skills}
\cvlanguage{{\footnotesize German}}{native speaker}{}
\cvlanguage{{\footnotesize Russian}}{fluent}{}
\cvlanguage{{\footnotesize Chinese}}{fluent} {HSK: B (7)\textcolor{white}{....}}
\cvlanguage{{\footnotesize English}}{fluent}{TOEFL iBT: 107\textcolor{white}{....}}
\cvlanguage{{\footnotesize Latin}}{good knowledge}    {Major Latinum\textcolor{white}{....}}
\cvlanguage{{\footnotesize Old Greek}}{good knowledge}    {Graecum\textcolor{white}{....}}
\cvlanguage{{\footnotesize Sanskrit}}{good knowledge}{}
\cvlanguage{{\footnotesize Swedish}}{basic knowledge}{}
\cvlanguage{{\footnotesize Danish}}{basic knowledge}{}
\cvlanguage{{\footnotesize French}}{basic knowledge}{}

 
\section{Computer Skills}
\cvlanguage{{\footnotesize Languages}}{Python, R, C{\ttfamily ++}, AWK, BASH (Unix Shell), PHP,
SQL, JavaScript}{}
\cvlanguage{{\footnotesize Formatting}}{\LaTeX{},
Microsoft Word, LibreOffice, HTML/CSS}{}
\cvlanguage{{\footnotesize Grafics}}{\hspace{0.0cm}PSTricks, GIMP, InkScape}{}
\cvlanguage{{\footnotesize
Operating Systems}}{\hspace{0.0cm}Linux, Windows}{}

\section{Publications}
\noindent\textit{{Im Erscheinen}}\par\nopagebreak\vspace{0.25cm}
\nopagebreak\noindent Geisler, H. and J.-M. List (forthcoming): \textbf{Beautiful trees on unstable ground. Notes on the data problem in lexicostatistics}. In: Hettrich, H. (ed.): \textit{Die Ausbreitung des Indogermanischen. Thesen aus Sprachwissenschaft, Archäologie und Genetik}. Reichert: Wiesbaden. \vspace{0.25cm}
\par
\noindent\textit{2014}\par\nopagebreak\vspace{0.25cm}
\nopagebreak\noindent List, J.-M., S. Nelson-Sathi, W. Martin, and H. Geisler (2014): \textbf{Using phylogenetic networks to model Chinese dialect history}. \textit{Language Dynamics and Change} 4.2. 222–252.\vspace{0.25cm}
\par
\nopagebreak\noindent List, J.-M. (2014): \textbf{Sequence comparison in historical linguistics}. Düsseldorf University Press: Düsseldorf.\vspace{0.25cm}
\par
\nopagebreak\noindent Mayer, T., J.-M. List, A. Terhalle, and M. Urban (2014): \textbf{An interactive visualization of cross-linguistic colexification patterns}. In: \textit{Visualization as added value in the development, use and evaluation of Linguistic Resources. Workshop organized as part of the International Conference on Language Resources and Evaluation}. 1-8.\vspace{0.25cm}
\par
\nopagebreak\noindent List, J.-M. and J. Prokić (2014): \textbf{A benchmark database of phonetic alignments in historical linguistics and dialectology.}. In: \textit{Proceedings of the Ninth International Conference on Language Resources and Evaluation}. 288-294.\vspace{0.25cm}
\par
\nopagebreak\noindent List, J.-M. (2014): \textbf{Investigating the impact of sample size on cognate detection}. \textit{Journal of Language Relationship} 11. 91-101.\vspace{0.25cm}
\par
\nopagebreak\noindent List, J.-M., S. Nelson-Sathi, H. Geisler, and W. Martin (2014): \textbf{Networks of lexical borrowing and lateral gene transfer in language and genome evolution}. \textit{Bioessays} 36.2. 141-150.\vspace{0.25cm}
\par
\noindent\textit{2013}\par\nopagebreak\vspace{0.25cm}
\nopagebreak\noindent Nelson-Sathi, S., O. Popa, J.-M. List, H. Geisler, W. Martin, and T. Dagan (2013): \textbf{Reconstructing the lateral component of language history and genome evolution using network approaches}. In: Fangerau, H., H. Geisler, T. Halling, and W. Martin (eds.): \textit{Classification and evolution in biology, linguistics and the history of science. Concepts – methods – visualization.}. Franz Steiner Verlag: Stuttgart. 163-180.\vspace{0.25cm}
\par
\nopagebreak\noindent Lopez, P., J.-M. List, and E. Bapteste (2013): \textbf{A preliminary case for exploratory networks in biology and linguistics: the phonetic network of Chinese words as a case-study}. In: Fangerau, H., H. Geisler, T. Halling, and W. Martin (eds.): \textit{Classification and evolution in biology, linguistics and the history of science. Concepts – methods – visualization.}. Franz Steiner Verlag: Stuttgart. 181-196.\vspace{0.25cm}
\par
\nopagebreak\noindent Geisler, H. and J.-M. List (2013): \textbf{Do languages grow on trees? The tree metaphor in the history of linguistics}. In: Fangerau, H., H. Geisler, T. Halling, and W. Martin (eds.): \textit{Classification and evolution in biology, linguistics and the history of science. Concepts – methods – visualization.}. Franz Steiner Verlag: Stuttgart. 111-124.\vspace{0.25cm}
\par
\nopagebreak\noindent List, J.-M. and S. Moran (2013): \textbf{An open source toolkit for quantitative historical linguistics}. In: \textit{Proceedings of the ACL 2013 System Demonstrations}. Association for Computational Linguistics. 13-18.\vspace{0.25cm}
\par
\nopagebreak\noindent List, J.-M., A. Terhalle, and M. Urban (2013): \textbf{Using network approaches to enhance the analysis of cross-linguistic polysemies}. In: \textit{Proceedings of the 10th International Conference on Computational Semantics -- Short Papers}. Association for Computational Linguistics. 347-353.\vspace{0.25cm}
\par
\noindent\textit{2012}\par\nopagebreak\vspace{0.25cm}
\nopagebreak\noindent List, J.-M. (2012): \textbf{Improving phonetic alignment by handling secondary sequence structures}. In: \textit{Computational approaches to the study of dialectal and typological variation. Working papers submitted for the workshop organized as part of the ESSLLI 2012}. \vspace{0.25cm}
\par
\nopagebreak\noindent List, J.-M. (2012): \textbf{LexStat. Automatic detection of cognates in multilingual wordlists}. In: \textit{Proceedings of the EACL 2012 Joint Workshop of Visualization of Linguistic Patterns and Uncovering Language History from Multilingual Resources}. 117-125.\vspace{0.25cm}
\par
\nopagebreak\noindent List, J.-M. (2012): \textbf{SCA. Phonetic alignment based on sound classes}. In: Slavkovik, M. and D. Lassiter (eds.): \textit{New directions in logic, language, and computation}. Springer: Berlin and Heidelberg. 32-51.\vspace{0.25cm}
\par
\nopagebreak\noindent List, J.-M. (2012): \textbf{Multiple sequence alignment in historical linguistics}\textbf{. A sound class based approach}. In: \textit{Proceedings of ConSOLE XIX}. 241-260.\vspace{0.25cm}
\par
\noindent\textit{2011}\par\nopagebreak\vspace{0.25cm}
\nopagebreak\noindent Holman, E., C. Brown, S. Wichmann, A. M\"uller, V. Velupillai, H. Hammarstr\"om, S. Sauppe, H. Jung, D. Bakker, P. Brown, O. Belyaev, M. Urban, R. Mailhammer, J.-M. List, and D. Egorov (2011): \textbf{Automated dating of the world's language families based on lexical similarity}. \textit{Current Anthropology} 52.6. 841-875.\vspace{0.25cm}
\par
\nopagebreak\noindent Nelson-Sathi, S., J.-M. List, H. Geisler, H. Fangerau, R. Gray, W. Martin, and T. Dagan (2011): \textbf{Networks uncover hidden lexical borrowing in Indo-European language evolution}. \textit{Proceedings of the Royal Society B} 278.1713. 1794-1803.\vspace{0.25cm}
\par
\noindent\textit{2010}\par\nopagebreak\vspace{0.25cm}
\nopagebreak\noindent List, J.-M. (2010): \textbf{Phonetic alignment based on sound classes}\textbf{. A new method for sequence comparison in historical linguistics}. In: \textit{Proceedings of the 15th Student Session of the European Summer School for Logic, Language and Information}. 192-202.\vspace{0.25cm}
\par
\nopagebreak\noindent Wichmann, S., E. Holman, A. Müller, V. Velupillai, J.-M. List, O. Belyaev, M. Urban, and D. Bakker (2010): \textbf{Glottochronology as a heuristic for genealogical language relationships}. \textit{Journal of Quantitative Linguistics} 17.4. 303-316.\vspace{0.25cm}
\par
\noindent\textit{2009}\par\nopagebreak\vspace{0.25cm}
\nopagebreak\noindent List, J.-M. (2009): \textbf{Sprachvariation im modernen Chinesisch}. \textit{CHUN -- Chinesischunterricht} 24. 123-140.\vspace{0.25cm}
\par
\nopagebreak\noindent List, J.-M. (2009): \textbf{Historische Aspekte der komparativen Methode}\ [Historical aspects of the comparative method]. Working paper.\vspace{0.25cm}
\par
\noindent\textit{2008}\par\nopagebreak\vspace{0.25cm}
\nopagebreak\noindent List, J.-M. (2008): \textbf{The validity of reconstruction systems}. Research Proposal.\vspace{0.25cm}
\par
\nopagebreak\noindent List, J.-M. (2008): \textbf{The puzzling case of Boshan tone sandhi}. Working paper.\vspace{0.25cm}
\par
\nopagebreak\noindent List, J.-M. (2008): \textbf{Resultativkomposita im Chinesischen}\ [Resultative compounds in Chinese]. Summary.\vspace{0.25cm}
\par
\nopagebreak\noindent List, J.-M. (2008): \textbf{Rekonstruktion der Aussprache des Mittel- und Altchinesischen}\textbf{. Vergleich der Rekonstruktionsmethoden der indogermanischen und der chinesischen Sprachwissenschaft}\ [Reconstruction of the pronunciation of Middle and Old Chinese. Comparison of reconstruction methods in Indo-European and Chinese linguistics]. Magister thesis. Freie Universität Berlin.\vspace{0.25cm}
\par
\noindent\textit{2007}\par\nopagebreak\vspace{0.25cm}
\nopagebreak\noindent List, J.-M. (2007): \textbf{Typology and linguistic reconstruction}. Working paper.\vspace{0.25cm}
\par
\nopagebreak\noindent List, J.-M. (2007): \textbf{The derivational character of the Chinese writing system}. Working paper.\vspace{0.25cm}
\par
\nopagebreak\noindent List, J.-M. (2007): \textbf{Osnovy rekonstrukcii srednekitajskogo i drevnekitajskogo jazykov}\ [Introduction to the reconstruction of Middle and Old Chinese]. Summary.\vspace{0.25cm}
\par
\nopagebreak\noindent List, J.-M. (2007): \textbf{Cóng Éwén de jiǎodù lái kàn pànduàn dòngcí "shì" de qǐyuán yǔ yǎnbiàn}\ {\hana 从俄文的角度来看判断动词“是”的起源与演变}\ [Origin and change of the Chinese copula shì from a Russian perspective]. Working paper.\vspace{0.25cm}
\par

\end{document}
