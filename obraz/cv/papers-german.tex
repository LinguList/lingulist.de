\noindent\textit{{Im Erscheinen}}\par\nopagebreak\vspace{0.25cm}
\nopagebreak\noindent Geisler, H. and J.-M. List (forthcoming): \textbf{Beautiful trees on unstable ground. Notes on the data problem in lexicostatistics}. In: Hettrich, H. (ed.): \textit{Die Ausbreitung des Indogermanischen. Thesen aus Sprachwissenschaft, Archäologie und Genetik}. Reichert: Wiesbaden. \vspace{0.25cm}
\par
\noindent\textit{2014}\par\nopagebreak\vspace{0.25cm}
\nopagebreak\noindent List, J.-M., S. Nelson-Sathi, W. Martin, and H. Geisler (2014): \textbf{Using phylogenetic networks to model Chinese dialect history}. \textit{Language Dynamics and Change} 4.2. 222–252.\vspace{0.25cm}
\par
\nopagebreak\noindent List, J.-M. (2014): \textbf{Sequence comparison in historical linguistics}. Düsseldorf University Press: Düsseldorf.\vspace{0.25cm}
\par
\nopagebreak\noindent Mayer, T., J.-M. List, A. Terhalle, and M. Urban (2014): \textbf{An interactive visualization of cross-linguistic colexification patterns}. In: \textit{Visualization as added value in the development, use and evaluation of Linguistic Resources. Workshop organized as part of the International Conference on Language Resources and Evaluation}. 1-8.\vspace{0.25cm}
\par
\nopagebreak\noindent List, J.-M. and J. Prokić (2014): \textbf{A benchmark database of phonetic alignments in historical linguistics and dialectology.}. In: \textit{Proceedings of the Ninth International Conference on Language Resources and Evaluation}. 288-294.\vspace{0.25cm}
\par
\nopagebreak\noindent List, J.-M. (2014): \textbf{Investigating the impact of sample size on cognate detection}. \textit{Journal of Language Relationship} 11. 91-101.\vspace{0.25cm}
\par
\nopagebreak\noindent List, J.-M., S. Nelson-Sathi, H. Geisler, and W. Martin (2014): \textbf{Networks of lexical borrowing and lateral gene transfer in language and genome evolution}. \textit{Bioessays} 36.2. 141-150.\vspace{0.25cm}
\par
\noindent\textit{2013}\par\nopagebreak\vspace{0.25cm}
\nopagebreak\noindent Nelson-Sathi, S., O. Popa, J.-M. List, H. Geisler, W. Martin, and T. Dagan (2013): \textbf{Reconstructing the lateral component of language history and genome evolution using network approaches}. In: Fangerau, H., H. Geisler, T. Halling, and W. Martin (eds.): \textit{Classification and evolution in biology, linguistics and the history of science. Concepts – methods – visualization.}. Franz Steiner Verlag: Stuttgart. 163-180.\vspace{0.25cm}
\par
\nopagebreak\noindent Lopez, P., J.-M. List, and E. Bapteste (2013): \textbf{A preliminary case for exploratory networks in biology and linguistics: the phonetic network of Chinese words as a case-study}. In: Fangerau, H., H. Geisler, T. Halling, and W. Martin (eds.): \textit{Classification and evolution in biology, linguistics and the history of science. Concepts – methods – visualization.}. Franz Steiner Verlag: Stuttgart. 181-196.\vspace{0.25cm}
\par
\nopagebreak\noindent Geisler, H. and J.-M. List (2013): \textbf{Do languages grow on trees? The tree metaphor in the history of linguistics}. In: Fangerau, H., H. Geisler, T. Halling, and W. Martin (eds.): \textit{Classification and evolution in biology, linguistics and the history of science. Concepts – methods – visualization.}. Franz Steiner Verlag: Stuttgart. 111-124.\vspace{0.25cm}
\par
\nopagebreak\noindent List, J.-M. and S. Moran (2013): \textbf{An open source toolkit for quantitative historical linguistics}. In: \textit{Proceedings of the ACL 2013 System Demonstrations}. Association for Computational Linguistics. 13-18.\vspace{0.25cm}
\par
\nopagebreak\noindent List, J.-M., A. Terhalle, and M. Urban (2013): \textbf{Using network approaches to enhance the analysis of cross-linguistic polysemies}. In: \textit{Proceedings of the 10th International Conference on Computational Semantics -- Short Papers}. Association for Computational Linguistics. 347-353.\vspace{0.25cm}
\par
\noindent\textit{2012}\par\nopagebreak\vspace{0.25cm}
\nopagebreak\noindent List, J.-M. (2012): \textbf{Improving phonetic alignment by handling secondary sequence structures}. In: \textit{Computational approaches to the study of dialectal and typological variation. Working papers submitted for the workshop organized as part of the ESSLLI 2012}. \vspace{0.25cm}
\par
\nopagebreak\noindent List, J.-M. (2012): \textbf{LexStat. Automatic detection of cognates in multilingual wordlists}. In: \textit{Proceedings of the EACL 2012 Joint Workshop of Visualization of Linguistic Patterns and Uncovering Language History from Multilingual Resources}. 117-125.\vspace{0.25cm}
\par
\nopagebreak\noindent List, J.-M. (2012): \textbf{SCA. Phonetic alignment based on sound classes}. In: Slavkovik, M. and D. Lassiter (eds.): \textit{New directions in logic, language, and computation}. Springer: Berlin and Heidelberg. 32-51.\vspace{0.25cm}
\par
\nopagebreak\noindent List, J.-M. (2012): \textbf{Multiple sequence alignment in historical linguistics}\textbf{. A sound class based approach}. In: \textit{Proceedings of ConSOLE XIX}. 241-260.\vspace{0.25cm}
\par
\noindent\textit{2011}\par\nopagebreak\vspace{0.25cm}
\nopagebreak\noindent Holman, E., C. Brown, S. Wichmann, A. M\"uller, V. Velupillai, H. Hammarstr\"om, S. Sauppe, H. Jung, D. Bakker, P. Brown, O. Belyaev, M. Urban, R. Mailhammer, J.-M. List, and D. Egorov (2011): \textbf{Automated dating of the world's language families based on lexical similarity}. \textit{Current Anthropology} 52.6. 841-875.\vspace{0.25cm}
\par
\nopagebreak\noindent Nelson-Sathi, S., J.-M. List, H. Geisler, H. Fangerau, R. Gray, W. Martin, and T. Dagan (2011): \textbf{Networks uncover hidden lexical borrowing in Indo-European language evolution}. \textit{Proceedings of the Royal Society B} 278.1713. 1794-1803.\vspace{0.25cm}
\par
\noindent\textit{2010}\par\nopagebreak\vspace{0.25cm}
\nopagebreak\noindent List, J.-M. (2010): \textbf{Phonetic alignment based on sound classes}\textbf{. A new method for sequence comparison in historical linguistics}. In: \textit{Proceedings of the 15th Student Session of the European Summer School for Logic, Language and Information}. 192-202.\vspace{0.25cm}
\par
\nopagebreak\noindent Wichmann, S., E. Holman, A. Müller, V. Velupillai, J.-M. List, O. Belyaev, M. Urban, and D. Bakker (2010): \textbf{Glottochronology as a heuristic for genealogical language relationships}. \textit{Journal of Quantitative Linguistics} 17.4. 303-316.\vspace{0.25cm}
\par
\noindent\textit{2009}\par\nopagebreak\vspace{0.25cm}
\nopagebreak\noindent List, J.-M. (2009): \textbf{Sprachvariation im modernen Chinesisch}. \textit{CHUN -- Chinesischunterricht} 24. 123-140.\vspace{0.25cm}
\par
\nopagebreak\noindent List, J.-M. (2009): \textbf{Historische Aspekte der komparativen Methode}\ [Historical aspects of the comparative method]. Working paper.\vspace{0.25cm}
\par
\noindent\textit{2008}\par\nopagebreak\vspace{0.25cm}
\nopagebreak\noindent List, J.-M. (2008): \textbf{The validity of reconstruction systems}. Research Proposal.\vspace{0.25cm}
\par
\nopagebreak\noindent List, J.-M. (2008): \textbf{The puzzling case of Boshan tone sandhi}. Working paper.\vspace{0.25cm}
\par
\nopagebreak\noindent List, J.-M. (2008): \textbf{Resultativkomposita im Chinesischen}\ [Resultative compounds in Chinese]. Summary.\vspace{0.25cm}
\par
\nopagebreak\noindent List, J.-M. (2008): \textbf{Rekonstruktion der Aussprache des Mittel- und Altchinesischen}\textbf{. Vergleich der Rekonstruktionsmethoden der indogermanischen und der chinesischen Sprachwissenschaft}\ [Reconstruction of the pronunciation of Middle and Old Chinese. Comparison of reconstruction methods in Indo-European and Chinese linguistics]. Magister thesis. Freie Universität Berlin.\vspace{0.25cm}
\par
\noindent\textit{2007}\par\nopagebreak\vspace{0.25cm}
\nopagebreak\noindent List, J.-M. (2007): \textbf{Typology and linguistic reconstruction}. Working paper.\vspace{0.25cm}
\par
\nopagebreak\noindent List, J.-M. (2007): \textbf{The derivational character of the Chinese writing system}. Working paper.\vspace{0.25cm}
\par
\nopagebreak\noindent List, J.-M. (2007): \textbf{Osnovy rekonstrukcii srednekitajskogo i drevnekitajskogo jazykov}\ [Introduction to the reconstruction of Middle and Old Chinese]. Summary.\vspace{0.25cm}
\par
\nopagebreak\noindent List, J.-M. (2007): \textbf{Cóng Éwén de jiǎodù lái kàn pànduàn dòngcí "shì" de qǐyuán yǔ yǎnbiàn}\ {\hana 从俄文的角度来看判断动词“是”的起源与演变}\ [Origin and change of the Chinese copula shì from a Russian perspective]. Working paper.\vspace{0.25cm}
\par
